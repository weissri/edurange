\section*{Activities}

main purpose of activities is to support faculty development; they need to incorporate security development into a wide range of courses,
and therefore they need more context. All activities are aimed at supporting this goal (and these outcomes).

We will bring together a wide range of experts who teach this stuff on a regular basis (plus their students) and we will put together
exercises that are specifically mapped to CS 2013 and specific implimentations of their security curriculum...catar a la carte menu to
fauclty that don't have that experience.

{\bf Ongoing Activities}

Monitoring and maintanece of EDURange cloud infrastructure as hosted
by Amazon. Support for mailing list and community building activities
(twitter, etc.?)

Ongoing support for EDURange infrastructure; possibly matched by Amazon


Student-based activities
 * help desk: direct support our faculty audience/clients; opportunity for professional development: super TA/student aide

 purpose is to twofold: close that last small gap in access (give faculty member comfort and confidence)

How do we control access? Faculty needs to commit themselves and students to participating in our assessment (see Section 5) then
on FCFS basis, we will offer a student 1, 2, or 3 times? We don't have capacity to help everybody in face-to-face premium mode, they
can get the mailing list asychronous support.

Platinum EDURange partners.

``1 out of 64 students went into teaching'': gives student opportunity to develop teaching/education technique...apprentice model, exposure
to a variety of professors and classroom environments...we don't teach grad students how to teach...this is an opportunity for practical
training and course and curriculum management, development, experiential learning for our student...valuable experience for undergrads, also
a variety of experience...they are not just pounding out code.. they act as a valuable external resource/expert to the faculty member they are
helping.

 * module development, multiple benefits to student researchers: pratical hands-on experience in systems and cloud and security, plus
   develop an appreciation for cybersecurity curriculum development as experssed in CS 2013.

Faculty-based activities 

 * design and run webinar (plan out curriculum) act as resource faculty
   audience; offering skeleton of an intro infosec course; cadre of instructors (panel of experts) will be given stipends to
   support their involvement
    * piazza
    * G+ youtube

 * money for workshops at SIGCSE where we demo EDURange for attendees --- this is place to get feedback; this activity
   feeds into our assessment and development (they give us ideas, VMs, exercises). Contact/face time target audience.

 * guide students and help design scenarios; major elements, maps
 elements to CS2013, and we will offer documentation (we are the best
 people to write this b/c we are faculty members...include notes for
 faculty on the scenario motivation, formative and summative
 assessments, description of how it maps to CS 2013, how to best use
 it, keywords/topics, learning objectives, provides default context
 and possible suggestions for integration into non-security courses.

development of new scenarios and exercises.
 - take other people's exercises, make accesible
 - create new exercises of the EDURange style

{Are we going to create student assessments?  We will do that for a small number
of exercises that we create.  Our goal is to create the framework for creating assessments.
 we will not be the only ones creating exercises in EDURange }



the webinar; the course template outline...make the point that
   the course is an example educational environment where EDURange can
   be mapped to or used.

a section on the EDURange ``help desk'' (another core expense item)



