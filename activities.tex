\section*{Activities}
\label{sec:activities}

The work contemplated in this proposal includes three major
activities: $(1)$ extending and enhancing the set of EDURange
scenarios (e.g., with appropriate documentation and pedagogical
advice), $(2)$ supporting faculty in their transition to using the
EDURange framework, and $(3)$ designing and running an open webinar to
disseminate a template outline for an introductory security course
(this template also serves as a default setting for applying
EDURange--based exercises).

Each activity and its supporting tasks are organized around the
efforts of a specific group of personnel (students, instructors,
PIs/researchers).  This section of the proposal describes the primary
research and training events.  Activities related to assessment are
described in Section~\ref{sec:eval}.

\subsection{Main Goal}

We stress that the main purpose of these activities is to support
instructor development; many faculty are in a position of needing to
develop and incorporate cybersecurity curriculum and exercises into a
wide range of courses -- often on relatively short notice.  The
audience for EDURange includes a wide variety of faculty members and
instructors with varying experience and opportunities for training
themselves in cybersecurity.  We have actively engaged with this
audience in a variety of workshops and meetings over the past five
years, and despite the variety in their backgrounds, they have
expressed a common need: the need for more context and expert support
in bringing cybersecurity material to their classrooms, tutorials, and
lab exercises.  In short, they express a need for help in
transitioning the wealth of available material to their specific
environment and translating this material into a suitable experience
for their students.  

This need is difficult to address during face-to-face workshops and
Birds-of-a-Feather sessions, and it is even harder to address when
faculty are faced with assimilating a large amount of available
cybersecurity training material unaided and unguided.  Faculty are
creative and talented, but they are in need of more context, timely
information, and relevant advice.  The activities in this proposal are
focused precisely on helping faculty achieve this goal and take the
final few but daunting steps between {\em knowing about} cybersecurity
exercises and {\em actually using} such exercises in their classrooms.

There is a gap between the potential value of cybersecurity exercises
and the actual benefit to students in practice.  Closing this gap and
achieving our goal through the activities outlined below can have a
direct impact on the quality of student learning, but our main focus
is on supporting and preparing faculty to design and deliver their own
student assessment activities (primarily via EDURange), not in
directly measuring student improvement.

%structure of each activity: who, what, why (when is in management plan) personnel, effort

\subsection{Activity I: Enhancing EDURange}

Our first major activity seeks to enhance the capabilities of EDURange
and sustain access to the infrastructure by removing barriers to
adoption, including the barrier of initial cost to faculty as they
first adopt it in their courses.  These efforts entail monitoring and
maintenance of the EDURange cloud infrastructure and various community
support mechanisms (mailing lists, help resources, etc.).  This
activity includes two significant sub-tasks.

\subsubsection{Sustainable Access to EDURange Cloud Infrastructure}

One major barrier to access that confronts faculty members is one of
limited resources.  EDURange is designed to eliminate as much of that
barrier as is practical, but running in the cloud still requires some
forms of delegation and management, particularly in accounting for
time and elastic resources allocated to each exercise scenario.  Part
of this proposal seeks support to sustain EDURange as a resource {\em
  while instructors become acquainted with EDURange and transition to
  their own educational Amazon grants}.  We have deliberately defined
an ``exit protocol'' to bolster a realistic sustainability plan.
EDURange is not meant to bear all the associated costs of a cloud
platform for all users in perpetuity.

\noindent {\bf Exit Protocol} \hskip 0.1in We seek to support
instructors in a way that eventually makes them independent, including
the ability to run EDURange scenarios from within their own Amazon
account or as funded by their own departments or institutions.  Our
plan stresses a significant amount of support early on and allows
instructors to become comfortable with using our EDURange scenarios as
delegated users within our Amazon account.  As instructors gain
comfort and familiarity with the technical details of the scenarios,
we will work with the instructor to transition them to an independent
Amazon account.

The driving factor behind this plan is a consideration of
sustainability: there is an obvious need to enable faculty and
instructors to operate scenarios on their own because no ``free''
service can support all users for an indefinite period of time.  Our
intent is that by the end of the period of the grant, we will have
created a community of independent EDURange users with a shared
experience and resources (our website, documentation, scenarios, etc.)
to help train others in the larger CS education community and new
faculty at their own institutions.  We see this as a great example of
sustainable capacity building because it is based primarily on
enabling instructors to look after themselves rather than creating a
perpetual dependency on the EDURange personnel.

\subsubsection{Module Development}

The second significant enhancement to EDURange concentrates on the
development and porting of cybersecurity scenarios to EDURange.  This
is primarily a student--driven activity.  

This activity accrues two benefits.  First, it augments the set of
scenarios available to users of EDURange.  Specifically, we will look
to port some existing cybersecurity exercises to EDURange.  The second
benefit is in providing our students not only the opportunity to get
practical hands-on experience in Computer Science topics like systems,
networking, cloud, and security, but also the opportunity to develop
an appreciation for cybersecurity curriculum development.  This theme
of benefit accruing to our students is a central consideration in the
activities described in this section. We seek ways to enhance their
educational experience that are not solely limited to software
development or programming.

\subsection{Activity II: Direct Faculty Support}

Our second major activity seeks to build our own capacity to offer
faculty significant and meaningful support, advice, and documentation
through mechanisms like peer instruction, debriefings, and guided
setup and semi-supervised delivery of EDURange scenarios. This
activity includes two significant sub-tasks.  The first is a
student--centric activity, and the second is a PI-centric activity.

\subsubsection{Semi-supervised Delivery (an EDURange ``Help Desk'')}

As mentioned above, faculty can be talented and motivated to learn
about cybersecurity and design courses and curriculum, but even those
with significant amounts of enthusiasm and awareness of cybersecurity
training resources prefer to have access to an informed ``second
opinion'', {\em especially} when seeking to incorporate live exercises
or demos from external organizations or other instructors.  To help
close that last small gap in access and give faculty members comfort
and confidence in deploying and using EDURange as an instructional
tool, we plan to develop a student--led ``help desk'' for EDURange.

This group of students, advised and supervised by the PIs, will be
available to provide live support to faculty who are using EDURange
scenarios for the first few times.  In essence, they will provide
instructors with semi-supervised delivery of lessons, tutorials, or
labs based on EDURange.  We know both that this model is possible and
practical because we have done it ourselves several times over the
past two years. We have prototyped this sort of interaction and are
ready to expand it.

\noindent {\bf Benefit to our students}: This direct support of our
faculty audience has several benefits, including the ability to more
closely and realistically assess how faculty are using EDURange and
any difficulties or benefits they perceive.  This kind of close
coordination thus has benefits to the project in terms of assessment
quality.  However, this kind of activity tremendously benefits our
students because it gives them the opportunity to practice and develop
their own teaching techniques. It gives them an exposure to a variety
of professors and classroom environments.  They act as a valuable
external resource and an expert peer to an instructor -- an experience
few undergraduate students ever have.  This form of experiential
learning should boost their own skill and confidence.

%MEL: I like this, but not sure we should SAY it in the proposal.  How
%do we control access? Platinum EDURange partners. Faculty needs to
%commit themselves and students to participating in our assessment
%(see Section 5) then on FCFS basis, we will offer a student 1, 2, or
%3 times? We don't have capacity to help everybody in face-to-face
%premium mode, they can get the mailing list asychronous support.

%``1 out of 64 students went into teaching'': gives student opportunity
%to develop teaching/education technique...apprentice model, exposure
%to a variety of professors and classroom environments...we don't teach
%grad students how to teach...this is an opportunity for practical
%%training and course and curriculum management, development,
%experiential learning for our student...valuable experience for
%undergrads, also a variety of experience...they are not just pounding
%out code.. they act as a valuable external resource/expert to the
%faculty member they are helping.

% MEL: Not just RTFM!
\subsubsection{Documentation}

The second significant task under Activity II is the development of
specialized documentation for each EDURange scenario.  Writing this
documentation is a PI--centric task because we are instructors {\em
  writing for other instructors} about our experience and expectations
related to curriculum material.  We can share how well each scenario
worked in the classroom environment and what topics it was useful for
introducing or assessing.  We can point out the pitfalls inherent in
some of the topics or technical details of each exercise.

In writing this documentation, we are not proposing to merely write
simple step-by-step instructions for running each scenario, but also
address very clearly and concisely a set of other important questions,
such as: what does scoring mean? What is the best way to use this
exercise in a particular class? What cybersecurity analysis skills are
being taught?  What are the knowledge and skill areas that can be
discussed before or after the scenario? How does this map to the CS
2013 curriculum?

Furthermore, we will include in each EDURange scenario description
information about what and how EDURange scores particular goals and
events within the exercise.  We will provide enough background
information and description to calculate a basic performance score,
and leave the weighting of each component to the particular instructor
using it.

We will provide notes and documentation for faculty on the scenario
motivation, learning objectives, keywords and topics, formative and
summative assessment suggestions, descriptions of how the exercise
maps to knowledge areas listed in CS2013, and suggestions for
integration into non-security courses.

Faculty are likely interested in the answers to these questions
because EDURange scenarios involve not only the technical details and
mechanics of using the AWS infrastructure, but also the security and
Computer Science curriculum issues that each scenario provokes.  We
plan to spend a significant amount of time documenting these
relationships.  An initial taste of this kind of information
(necessarily brief, and far from complete) is given in
Section~\ref{sec:modules} for each EDURange exercise description.

\subsection{Activity III: Webinar and Workshops}

Our third major activity focuses on a specific set of
community--building exercises that are directly linked to two
components of this proposal: $(1)$ providing an example course outline
to act as a default ``setting'' for all the EDURange scenarios, and
$(2)$ acquiring assessment of and feedback on EDURange itself.  This
is a faculty--centric activity and involves the organization of both
virtual meetings and face--to--face workshops.  Our plans for these
events are opportunistic in reducing costs in that they take advantage
of our existing connections with our community and co-location at
independent events.

\subsubsection{A Short Course ``Webinar''}

Many faculty face the challenge of formulating a course design and
syllabus; it is a central part of our job.  When faced with the task
of creating a security course, or even of finding a set of security
course modules to incorporate into a non-security course, some
guidance is valuable, even if it is a rough course outline or template
or categorized list of readings. Such artifacts, simple as they are,
can be quite valuable because they impose a basic intellectual
structure on an expansive and intricate topic.  We plan to present the
outline for an introductory security course to an audience of
instructors. The audience can then use this outline to jumpstart their
own course planning.

Many proposals for building cybersecurity capacity via the training or
professional development of faculty can suffer from the practical
costs and difficulties of arranging continued and costly
face--to--face meetings.  We recognize these challenges, and in
response to the CS2013 panel findings on cybersecurity suggesting the
design of a few MOOC-style courses, we have designed an online
webinar, the purpose of which is to present the outline of a
``typical'' introductory information security course outline in a
short course style.

Our intent is to conduct hour-long online video chats with the
EDURange community we have identified over the past few years
(attendees at our workshops and BoFs, other colleagues from
cybersecurity venues and groups).  We plan a total of 12..15 of these
sessions, which will take place via Google Plus video Hangouts (or
other suitable mechanism) and be streamed and saved to YouTube, and
linked from our website \url{www.edurange.org}.  We plan to use the
Piazza platform to continue and facilitate the discussions arising
from the video chats or clarify issues raised by our presentation. The
course outline itself will be available via our EDURange github
account.  The bulk of the effort will be concentrated in preparatory
work of the design and scripting of these sessions.

Each session will present a component (i.e., topic) of a 10 week
course and split time between the presentation by the lead
facilitators and discussion among the attendees.  We assume that each
component or course topic would be addressed by two to three classroom
lectures per week.

The course itself is based on a fusion of our own introductory
security courses.  Although we are proud of our work, we do not claim
to have a monopoly on the ``best'' cybersecurity course.  We have
invited several other facilitators to join us in designing the course
outline and helping guide the discussion during the video chat
sessions.

\noindent {\bf Benefit to the community} \hskip 0.1in We point out
that our purpose is simply to provide a context in which EDURange
exercises can be logically demonstrated; hopefully some of our
audience of instructors finds the course syllabus to be useful as
well, even as a place for them to start their own planning or course
development.  With this webinar component of the proposal, we are
essentially offering an expert panel as a resource to instructors
seeking to (re)develop cybersecurity topics or courses.

\subsubsection{Workshops}

A vital part of this proposal is the assessment of how well these
activities are supporting faculty in their vocation of delivering
cybersecurity education.  Our specific assessment plans are discussed
in Section~\ref{sec:eval}, but a critical precondtion of the
assessment is obtaining feedback from instructors that use EDURange.
For the past several years, we have held workshops at Computer Science
educational conferences where faculty can experience an EDURange
scenario.  We plan to continue holding such workshops.

This activity is of tremendous value because it feeds into both our
assessment mechanisms and our development priorities for EDURange
(both the infrastructure and the scenarios).  We have found that
workshop participants are a wonderful font of ideas for exercises,
scoring strategies, and other improvements.  Workshops serve as
valuable face--to--face time with our target audience and allow us to
recruit further audience members by giving them firsthand experience
with us and the infrastructure.

\subsection{Additional Value of These Activities}

The EDURange environment is meant to provide an efficient mechanism
for students to immediately engage with a live exercise.  EDURange
minimizes and eliminates the need for both faculty and students to
deal with logistics and administrative setup work for labs, tutorials,
or other ``hands-on'' cybersecurity experiences. 

This proposal brings together a wide range of experts who teach
cybersecurity on a regular basis in a variety of settings.  Their
students are active participants in all facets of the project.  One
important potential contribution of EDURange is that it can offer one
model of how the CS2013 principles, knowledge, and skills related to
cybersecurity can be implemented in actual exercises and lessons. We
plan to actively map these relationships.
