\section{SISMAT: An Immersive Approach to Infosec Education}
\label{sec:intro}

A significant cybersecurity workforce would provide a strong pillar
for the domestic high-tech industry.  The nature of these jobs demands
that they remain in the United States for the long term, and they
would directly support efforts to introduce information technology
into various parts of our society (e.g., the health care and energy
industries) in a secure and reliable fashion.  Without a commitment to
educating such a workforce, it is ill-advised to simply hire such a
workforce into existence.  Yet, many traditional educational options
are not well-suited to producing either the volume or quality of
professionals that forecasts~\cite{krebs} claim are needed.  Preparing
a large cybersecurity workforce seems to be a national priority, but
emerging viewpoints~\cite{cooper2009sigcse} suggest that most efforts
to date have not been as effective as government and industry seem to
need~\cite{locasto2011cacm}.  GovTech.com suggests that there is a
``lack of faculty at the university level who can teach cyber-security
beyond its ``soft side'' including policy and
analysis.''\footnote{\url{http://www.govtech.com/security/Cyber-Challenge-Work-Force-041811.html}}

This collaborative proposal advocates for a specialized, intense, and
immersive training experience for undergraduates; this experience
helps nurture their interest in information security at an early and
critical period in their professional development.  The proposed
program builds on a previous pilot program (2008-2010) and the pilot
program's follow-on effort (2010-present) to demonstrate that this
particular model (see Section~\ref{sec:prior} for a description) can
work well in giving undergraduates a boost into a cybersecurity
career.  This proposal seeks to expand the program's scale by offering
it on both the East and West Coast each year. This new direction
actually represents one of the goals of our previous CCLI Type 1
proposal: ``our project will be a model that may be adapted to work in
other regions of the country at other institutions.''  This
partnership fulfills this goal and represents a collaborative effort
between three different colleges and universities: Dartmouth College,
Lewis \& Clark College, and the University of California Davis.  The
relationships and expertise involved represent a unique mix of
cybersecurity and education skills and experience.

%% THE GAP
\subsection{Motivation}

From our point of view, far too few workers are adequately trained
mostly because traditional educational mechanisms lack the resources
to effectively train large numbers of experienced, knowledgeable
cyber-security specialists.  The incredibly diverse cyber-security
needs of both industry and government complicate matters: operational,
analytical, and strategic technology roles span all parts of these
organizations.  Although professional certification courses exist, and
the NSA has designated many college and university cyber-security
programs as Centers of Academic Excellence (CAE), in reality, only a
small number of quality educational programs are funded, equipped, and
willing to quickly educate significant numbers of information security
professionals~\cite{spaf}.

University education can serve a pivotal role in providing the core
skills necessary for a professional workforce to be adaptive to a
threat that is hyper-adaptive.  Plans for training cyber-security
workers should focus on educating a new workforce to complement
efforts at certifying existing workers.  Unfortunately, while the
university environment seems like an ideal place for strategic
cybersecurity education, several obstacles conspire to frustrate the
goals of such educational programs.

It is not easy for undergraduates to gain a foothold into the world of
cybersecurity professionals.\footnote{While obstacles like
  certification and security clearances are present, we focus more on
  the systemic issues of education rather than procedural issues
  involved in obtaining employment.}  Because security is an
enterprise- or system-wide concern, many organizations look for
seasoned security professionals who by definition have expertise
across many different technologies (which helps them understand why
complex systems fail in unexpected or malicious ways).  Unfortunately,
undergraduates typically do not have this kind of extensive
experience, and so even if they have a deep interest in information
security, it can be difficult for them to learn about the topic.
Self-instruction can be confusing and time-consuming for the unguided
novice.  Many undergraduate Computer Science programs do not have any
security curriculum (and are not SFS scholarship participants or NSA
CAEs).  Waiting to attend an information security graduate program
typically means delaying the fulfillment of a student's interest in
cybersecurity until after they have completed a BSc degree.  Although
highly motivated employers may aggressively recruit undergraduate
students and pay to send them to industry infosec training or
certification courses, it seems unclear whether such courses teach the
analytical skills necessary for strategic cybersecurity thinking
(i.e., the kind of thinking expected of a college education, and the
kind of thinking in increasingly high demand to adapt to a
hyper-adaptive adversary).

In addition, teaching security analysis skills to undergraduates
presents an imposing challenge.  An information security curriculum
can be difficult to formulate and weave into existing classes,
particularly for instructors with little background in the topic.
Most existing security curriculum and exercises focus on teaching
students basic, secure coding practices ({\it e.g.}, input validation,
bounds and error code checking) and typically lack an interactive,
experiential learning component.

\subsection{Proposal Focus: Enhancing the Delivery of Cybersecurity Education}

We propose to complement existing cybersecurity education programs
through an expansion of the SISMAT (Secure Information Systems
Mentoring and Training, NSF CCLI 0941836) program.  SISMAT helps fill
the gaps identified above by offering an alternative educational
environment that traditional undergraduate programs and other vehicles
for cybersecurity education are not quite structured to offer.  We
believe that SISMAT can help enhance the delivery of cybersecurity
education in the United States.  Our past experience with SISMAT has
been very encouraging.  Several of the faculty mentors have been
inspired to teach courses based on what they have learned from their
participation.

SISMAT provides a unique comprehensive experience for undergraduates
that enables these students to develop their information security
analysis skills.  This intense, immersive experience is aimed at an
otherwise under-served population of students from liberal arts
undergraduate institutions: schools that rarely have the faculty
expertise and facilities to undertake information security education
or research projects.

\subsection{Roles}

Designing, coordinating, and running a program as comprehensive as
SISMAT is no small task, especially as we ramp up to include two
parallel instances on the East and West Coasts.  We formulated a group
of deeply engaged educators and researchers with a rich variety of
backgrounds in information security to help deliver the program.  PI
Bratus will oversee the project and lead selected instructional
sessions.  Dr. Locasto will supervise and co-lead instructional
sessions on both the East and West Coast.  He will assist the PIs in
curriculum development and dissemination.  Co-PI Mache will oversee
the delivery of the West Coast version of SISMAT, and they will work
with Dr. Locasto and co-PI Peisert to coordinate the lectures, labs,
and seminar material.  They will also interface with Dr. Nilsen, who
will provide his expertise to serve as the project's evaluator.  He
will stay in close contact with the PIs and advise them on an ongoing
basis to ensure that the project goals are met and that the assessment
activities take place.  This evaluation role is critical because it
provides a way of understanding how effective the project has been at
nurturing a security analysis skill set among participants.

\subsection{Proposal Components}

Our proposal helps address the STEM learning components identified in
the TUES solicitation (we focus on these proposal components even in
the context of a submission to SFS because TUES is the program that
SISMAT is growing from and shares many of the objectives of the SFS
Capacity Building track).  Table~\ref{table:TUES} lists these TUES
themes and the matching proposal components.  Over the past four
years, we have created a community of students and researchers that
are engaged on this topic, and who will help further develop its reach
and recruit student participants and internship contacts.  In fact,
two of the faculty members on this proposal are SISMAT mentors who
became very engaged as a result of their students attending SISMAT and
their interest in information security education.  We see this as an
example of the type of engagement that SISMAT naturally fosters.  We
are constantly refreshing this community through BoFs at ACM SIGCSE,
forming professional contacts through our recruiting activities, and
leveraging the current SISMAT faculty community through the SISMAT
faculty email list.

\begin{table}[ht]
\begin{center}
\begin{tabular}{| p{2in} | p{3.5in} |}
\hline
  {\bf TUES Solicitation Component} & {\bf Proposal Item} \\
\hline\hline
  \raggedright {\bf Implementing Educational Innovations} & The application of the ``Hacker Curriculum'' to developing lectures and labs; the SISMAT program as a comprehensive educational experience (seminar, internship, research project)\\
  \hline
  \raggedright {\bf Creating Learning Materials and Teaching Strategies} & 
   Creating and maintaining the SISMAT lab manual, YouTube videos of labs and lectures\\
\hline
\eat{
  \raggedright {\bf Research on Undergraduate STEM Teaching and Learning} & ``Grand Challenge'' joint research project provides a research-based learning experience for students\\
  \hline} %end comment
  \raggedright {\bf Assessing Learning and Evaluating Innovations} & 
  Continuous involvement of our evaluator, Dr. Nilsen\\
  \hline
  \raggedright {\bf Developing Faculty Expertise} & Community outreach 
   through SIGCSE BOFs, the SISMAT faculty email list, synergistic proposals for a SISMAT-like program for faculty, faculty participation in the grand challenge research project, sharing curriculum resources with faculty\\
  \hline

\end{tabular}
\end{center}
\caption{{\em How the SISMAT Project Meets TUES Needs}. These goals also align with the SFS Capacity Building track.}
\label{table:TUES}
\end{table}

\subsection{Goal} 

% Jen's idea... i like it! -MEL
This proposal does not merely carry forward current SISMAT efforts;
rather, it represents an aggressive, ambitious expansion of SISMAT to
both increase the scale and retool a key component: the mentored
research project by transforming it from an individual project to a
joint collaborative one between all participants (participants still
have the option of undertaking their own individual project instead).

Inspired by NSF's ``CS10K'' effort to jump-start AP Computer Science
at 10,000 high schools, our overall goal is no less than jump-starting
hands-on, technical information security experience at about 50 to 100
undergraduate programs\footnote{We recognize that faculty with
  interests and expertise in information security already exist in
  some of this target population; we do not claim to be the first or
  only way such a boost can be achieved, but we believe that SISMAT
  can serve as a very efficient vehicle for helping achieve this
  goal.} over the next four years (roughly 30 students per year
from 2013 to 2016 as reflected in our proposed budget -- about 120
student-faculty pairs).

\subsection{Planned Outcomes}
\label{ssec:outcomes}

We seek the following specific outcomes:\\

\noindent {\bf Outcome 1: Expansion of the core SISMAT program} --- We
will expand SISMAT from one program instance per year to two instances
with one centered on the East Coast at Dartmouth College and the other
centered on the West Coast at Lewis and Clark.  This expansion will
double our yearly capacity to train and mentor participants.  This
outcome most directly supports our goal of helping nurture information
security expertise at about 100 undergraduate institutions over the
next four years.

\noindent {\bf Outcome 2: Successful transition of participants to
  cybersecurity jobs and post-graduate education} --- We expect that
program participants will continue on to graduate degrees in
information security or jobs in cybersecurity.  We plan to track their
progress and professional careers to help measure the impact that
SISMAT has had in producing (in a small but significant way) a part of
the future cybersecurity workforce; this tracking will be in the form
of follow up surveys (we currently conduct these with all past
participants) and maintaining a LinkedIn group.  While we are not
proposing to address the supply needs of thousands of cybersecurity
professionals over the next few years, we are proposing to attract
talented students to the field and teach them to become the next
generation of teachers and researchers.  We posit that this kind of
contribution is a tangible way to help build the critical mass for a
generation of cybersecurity professionals.

\noindent {\bf Outcome 3: An impact on the information security
  community by addressing a ``grand challenge'' problem each year} ---
Having about 60 students and faculty per year available to undertake a
research project is a powerful opportunity.  Rather than individual
experiences, we plan to undertake a joint applied research project
each year with program participants (participants will retain the
option to undertake their own research project if they wish).  By
leveraging the collective energy and attention of a significant number
of professors and students, each year we can address an important
strategic challenge facing the information security community.  This
collection and focus of effort would otherwise only be available
through an industry consortium or a free software project community.
No academic research labs or groups that we know of (and only a few
companies) could marshal an equivalent workforce focused on a single
problem.  We provide examples of such challenge problems in
Section~\ref{ssec:grand}.
%Google summer of code on steriods for security

\noindent {\bf Outcome 4: Information security resources} --- We will
continue to document and share the outcomes of the SISMAT program,
including the lessons, lectures, and labs.  We will do so through the
SISMAT wiki, the hackercurriculum.org website, YouTube videos of
SISMAT lectures, and a dynamically-updated lab manual.  Our aim in
producing this outcome is to supply our educational partners with a
menu of resources they can easily incorporate into their own core
undergraduate Computer Science classes or into specific information
security elective courses they may teach in the future.  This outcome
is a central part of our transition plan: by producing these
resources, we hope to help partner institutions become more
self-sufficient and confident in their cybersecurity expertise and
knowledge.  Over the lifetime of the program, we will disseminate
these resources by publishing in Computer Science education journals
and conferences, holding BoFs and workshops at these conferences, and
maintaining the SISMAT-faculty email list.  Our experience and
involvement in the Computer Science education community will help us
engage with this audience and their future students.

\subsection{Faculty Involvement}
\label{ssec:fac}

Before moving on to describe the SISMAT program itself and our plans
for expanding it, we stress that this proposal focuses primarily on
means and methods for undergraduate education, not professional
development, training, or education for faculty members.  We do
believe, however, that a SISMAT program specifically for faculty might
be of tremendous benefit (and we have heard such suggestions from many
interested faculty during our student recruiting efforts).
Nevertheless, the focus of this proposal is on student training.
Faculty training and professional development can be addressed through
complementary capacity building programs.

While SISMAT does not address the need to train faculty members
directly, SISMAT naturally benefits student participants' faculty
mentors in a number of implicit and indirect ways.  We believe that
the engagement and support of a faculty mentor makes a significant
difference for SISMAT student participants.  SISMAT encourages such
engagement in a number of ways.

We engage faculty mentors by asking them to work with the student on a
research project or independent study having to do with security (the
faculty member also has to provide a letter of support for the
student's application to the program).  Secondarily this research
project gives the faculty member an opportunity to learn more about a
security topic that interests both them and the student.  In addition,
some faculty have designed their own course around the SISMAT
materials and curriculum.  We regularly hold BoF sessions at Computer
Science education conferences to provide a forum for professors
interested in information security.  Finally, we maintain a both a
LinkedIn group (for all SISMAT participants) and an email list of all
past SISMAT faculty participants as a forum for discussing security
issues, passing on interesting job opportunities for students, and
disseminating SISMAT curriculum materials.
