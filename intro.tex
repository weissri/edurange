%\documentclass[11pt]{report}
%\usepackage{times}
% \usepackage{fullpage}
%\usepackage{url}

%\newcommand{\comment}[1]{}

%\begin{document}


\section{Introduction: Helping Faculty Teach Analysis Skills}
\label{sec:intro}

% the general problem
According to published reports by the SANS Institute and other 
groups~\cite{defensenewsshortage}, the US faces a major shortage of 
security professionals to defend our information infrastructure from
attack.  In recognition of this need, security has been included as
a core topic in the new 
IEEE/ACM CS2013 Curricula~\cite{acmcurriculum}.   
This was re-iterated in the ACM report ``Toward Curricular Guidelines for Cybersecurity''.
The latter report also calls for MOOCs for faculty to provide training in cybersecurity.
At educational conferences such
as SIGCSE and regional CCSC conferences, we are seeing a growing interest 
in cybersecurity among faculty who do not have expertise in this area.
We propose to address the needs of two groups of faculty: those who have
significant experience in cybersecurity and those who do not.
It is still
rare for schools to offer cybersecurity as an elective
because of the tight 
constraints of the Computer Science curriculum, most schools 
do not have the luxury of offering a separate class in cybersecurity.  Thus, 
the first step is to integrate it into other classes both at the upper and lower divisionial
levels.  It was also reported at the panel .. at SIGCSE 2014 that only one Ph.D. in Computer
Security in 2012 joined the ranks of faculty who teach.

Unfortunately, most students 
in computer science receive very little training in security as
undergraduates and lack the necessary skills 
for these jobs.

% What problem are we solving?
 Professors and instructors face a significant number of
obstacles to providing students with realistic enough environments
where students can safely develop these skills and mindset.  Although
a variety of curricular resources exist, most of them focus on very specific
toolsets or traditional, well-understood types of vulnerabilities and
exploits.  Among the few ``interactive'' scenarios that do exist, documentation
 often  does not adequately explain the
security implications of the scenarios that students participate in.  Even faculty
with some experience in systems administration would need to 
spend a significant amount of time configuring the exercise, and it is 
generally not flexible enough to be adapted to their needs or wishes.
In addition, we plan to offer a webinar for faculty on teaching an introductory 
Computer Security class.
\eat{
In the report {\it Toward Curricular Guidelines for Cybersecurity} \cite{ACM_cyber_2013},
the first recommendation for the NSF is to provide MOOCS to assist teaching of the 
fundamentals as prescribed by the CS2013 exemplars.  This proposal describes a webinar 
that we plan to hold  which we believe would have similar goals and would be more effective.}

For students, the process of analyzing the security
properties of computer systems presents a variety of difficulties.
Acquiring the mindset necessary to analyze and debug systems -- and
thereby understand how they can't be trusted, requires a level
of engagement that most typical computer science (CS) undergraduate assignments lack and
most typical information security courses do not have time for.  
Students are typically rewarded for following a formulaic trail: a sequence
of commonly used tools.  In addition, assessing students' analysis skills is  difficult 
problem.


The {\bf EDURange framework} for cybersecurity exercises has been successfully demonstrated 
in several venues: security classes at two liberal arts colleges, a graduate-level security class, and
two workshops for faculty at major conferences.  There has been strong interest on the part of
faculty participants in using this platform in their classes.  The design goals are:
\begin{packenum}
\item engaging for the students
\item easy to assess by faculty
\item easy to configure
\item exciting for the faculty, so that they will be engaged and will recommend them
  to colleagues
\end{packenum}

However, one of the potential barriers to distributing EDURange to them is the hosting environment.
EDURange is currently hosted on
Amazon's AWS/EC2 virtual cloud environment.  This has made it very easy for the PIs to use it in 
workshops anywhere without having to worry about installing special software.  There are no plugins;
the participants only need an ssh client, e.g. PuTTY on Windows.  The current 
model for distributing
exercises to other faculty requires that they create an Amazon account and potentially require their students
to also sign up for accounts.  We propose lowering the barrier for entry in three ways:

{\bf First}, by creating an umbrella account in 
which we could enroll faculty and their classes.  This facilitates distribution of EDURange
and the abililty to provide direct 
technical support.  

{\bf Second},
we propose to provide a ``help desk'' which could guide faculty through any difficulties in setting up 
exercises and using the AWS interface.  From one of our recent workshops, we saw that 
some faculty felt that they needed more than a three-hour workshop to prepare them for using this
environment.  We may need to help faculty who are not familiar with AWS or the EDURange
framework  to set up exercises 
for their classes.  Our solution to this problem is  to have a team of undergraduates who would staff 
an online help desk
and answer questions about how to launch VM instances, run scripts and modify existing exercises or use them as 
templates for new exercises.

{\bf Third}, we will run an online course/webinar for faculty using EDURange to teach an introductory cybersecurity
course.  We have already talked with other faculty who teach cybersecurity, and they are interested in 
working with us to share their syllabuses and help deliver this online course.  We have talked with 
Corrinne Sande, a PI for Cyber Watch West, and we would be able to offer this webinar through their 
facilities.
Instructors who sign up for this course would learn
to use EDURange to enhance their curriculum and create new curriculum for their classes, which could range from 
a standard introductory
security class to including some 
security in other classes, e.g. Networking, OS, Database systems, Programming Languages.
We will provide EDURange for free to the instructors and their students over the next five years.

%Fourth, we will develop the EDURange scoring engine that will allow faculty to use EDURange exercises
%to assess student learning.

% does this come before the EDURange stuff above?
Our vision is that EDURange would provide a vehicle for all Computer Science faculty to {\em address the goals of 
the IEEE/ACM CS2013
Curricula} report by translating the Information Assurance and Security (IAS) core concepts and outcomes into
concrete examples.  There are three interconnected components of this proposal: further developing the
EDURange framework to enhance its capabilites for dissemination and assessment, provide support for faculty development
through webinars, and provide a research opportunity for students to work on developing EDURange and work
with faculty who are using it in their classrooms.
\paragraph{Outcomes}
The outcomes of this work will be:
\begin{enumerate}
\item EDURange will be used in 100 classes over five years at institutions other than the home institutions of
  the PIs.
%\item Enhance the EDURange framework to improve its assessment capabilities MEL: may be a distraction for this proposal
\item Faculty will be able to assess their students on topics in the CS2013 report using EDURange and 
  associated materials.
\item EDURange and associated activites will enhance teaching capabilites and enhance faculty expertise.
\item EDURange will produce a concrete realization of several of the IAS-related Knowledge Units
  with respect to cybersecurity concepts, skills and learning outcomes.
\end{enumerate}



\subsection{Analysis and the Security Mindset}
As security educators, we are motivated by the desire to teach analysis skills and concepts
to our students.
Analysis skills are fundamental to the security mindset.  The security
mindset implies the ability to understand a system both from the
standpoint of a builder and an attacker.  Thus, the security mindset
provides the conceptual underpinnings for a student to reason in both
defensive and offensive situations.  One aspect is failure modes,
i.e. how can systems fail and be made to fail.  These are things that
a defender or attacker needs to keep in mind.  Another aspect is
questioning and verifying assumptions.  This is particularly relevant
for the defender.

We want to change how security is taught in the classroom, what is taught, improve 
faculty confidence to teach it, and make CS2013 a living document.  We want our students to 
have the opportunity to go beyond 
'Familiarity' to 'Usage', to be able to apply the concepts in realistic situations.


\noindent from a CS2013 exemplar: ``The goal of this course is to introduce students to the challenges, approaches, and techniques 
for implementing, deploying, and maintaining secure computing systems and networks. The rationale behind this 
course is to meet the
need for information assurance content within the CS curriculum at the same time as tailoring to the diversity of
potential career paths in the program's student population.''
