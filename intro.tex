%\documentclass[11pt]{report}
%\usepackage{times}
% \usepackage{fullpage}
%\usepackage{url}

%\newcommand{\comment}[1]{}

%\begin{document}


\section{Helping Students Acquire Security Analysis Skills}
\label{sec:intro}

% the general problem
According to published reports by the SANS Institute and other 
groups~\cite{defensenewsshortage}, the US faces a major shortage of 
security professionals to defend our information infrastructure from
attack.  Unfortunately, most students 
in computer science receive very little training in security as
undergraduates and lack the necessary skills 
for these jobs.  In recognition of this need, security has been included as
a core topic in the new 2013 
IEEE/ACM Curricula~\cite{acmcurriculum}.   At educational conferences such
as SIGCSE and regional CCSC conferences, we are seeing a growing interest 
in cybersecurity among faculty who do not have expertise in this area.
We propose to address the needs of two groups of faculty: those who have
significant experience in cybersecurity and those who do not.
It is still
rare for schools to offer cybersecurity as an elective
because of the tight 
constraints of the Computer Science curriculum, most schools 
do not have the luxury of offering a separate class in cybersecurity.  Thus, 
the first step is to integrate it into other classes both at the upper and lower divisionial
levels.


% What problem are we solving?
 Professors and instructors face a significant number of
obstacles to providing students with realistic enough environments
where students can safely develop these skills and mindset.  Although
a variety of curricular resources exist, most of them focus on very specific
toolsets or traditional, well-understood types of vulnerabilities and
exploits.  Among the few ``interactive'' scenarios that do exist, documentation
 often  does not adequately explain the
security implications of the scenarios that students participate in.  Faculty
spend a significant amount of time configuring the exercise, and it is 
generally not flexible enough to be adapted to their needs or wishes.
Students are typically rewarded for following a formulaic trail: a sequence
of commonly used tools.  

For students, the process of analyzing the security
properties of computer systems presents a variety of difficulties.
Acquiring the mindset necessary to analyze and debug systems -- and
thereby understand how they can't be trusted, requires a level
of engagement that most typical computer science (CS) undergraduate assignments lack and
most typical information security courses do not have time for.  **should we say something
here about trust and trustworthiness, which is in CS2013? **

The EDURange framework for cybersecurity exercises has been successfully demonstrated 
in several venues: security classes at two liberal arts colleges, a graduate-level security class, and
two workshops for faculty at major conferences.  There has been strong interest on the part of
faculty participants in using this platform in their classes.  
One of the potential barriers to distributing EDURange to them is the hosting environment.
EDURange is currently hosted on
Amazon's AWS/EC2 virtual cloud environment.  This has made it very easy for the PIs to use it in 
workshops anywhere without having to worry about installing special software.  There are no plugins;
the participants only need an ssh client, e.g. PuTTY on Windows.  However, the current 
model for distributing
it to other faculty requires that they create an Amazon account and potentially require their students
to also sign up for accounts.  We propose to lower the barrier for entry with an umbrella account in 
which we could enroll faculty and their classes, so that we could distribute it more easily.  In addition,
we propose to provide a ``help desk'' which could guide faculty through any difficulties in setting up 
exercises and using the AWS interface.

In addition, we see that there will be a need to help faculty initially set up exercises 
for their classes.  We
would like to have a team of undergraduates who would staff an online help desk
and answer questions about how to run scripts and modify existing exercises or use them as 
templates for new exercises.

Third, we will run an online course for faculty using EDURange to teach an introductory cybersecurity
course.  We have already talked with other faculty who teach cybersecurity, and they are interested in 
working with us to deliver this online course.  Instructors who sign up for this course would learn
to use EDURange to create curriculum for their classes, which would range from a structured introductory
security class to including some 
security in other classes, e.g. Networking, OS, Database systems, Programming Languages.
We will provide EDURange for free to the instructors and their students over the next five years.

Fourth, we will develop the EDURange scoring engine that will allow faculty to use EDURange exercises
to assess student learning.


\subsection{Outcomes}
\begin{enumerate}
\item Disseminating EDURange to 100 faculty, who will use it in their classes.
%\item Enhance the EDURange framework to improve its assessment capabilities MEL: may be a distraction for this proposal
\item Enhance the EDURange implementation to improve the ability to disseminate it.
\item Create a collection of 10 exercises that would span an introductory security course.
\end{enumerate}

\subsection{The Security Mindset}

Analysis skills are fundamental to the security mindset.  The security
mindset implies the ability to understand a system both from the
standpoint of a builder and an attacker.  Thus, the security mindset
provides the conceptual underpinnings for a student to reason in both
defensive and offensive situations.  One aspect is failure modes,
i.e. how can systems fail and be made to fail.  These are things that
a defender or attacker needs to keep in mind.  Another aspect is
questioning and verifying assumptions.  This is particularly relevant
for the defender.
