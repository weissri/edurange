\section*{Assessment Plan}

Answer the question: how will we demonstrate this enhanced value? Raw numbers,
plus qualitative application of CS 2013. Demonstrate how much people use EDURange
now that they have open access to it.

Assessment Plan -- this section will have 2
   sub-parts. One on raw metrics of number of users, students, profs, classes, etc. etc. 

NB: any discussion of the scoring engine in EDURange probably does not belong here; that is
 an EDURange I concern, not for this proposal.

The second is on how well the EDURange infrastructure and scenarios
map to the skills and knowledge of security throughout the CS 2013
undergrad curriculum. Assessment plan describes the assessment
mechanisms, procedures, and practices


First pillar: usage stats, drawn from EDURange and Amazon infrastructure. catelogue
 frequency and amount of usage per prof or class. How much was EDURange used in particular
 kinds of classes? Is there a difference in usage given the type of course (e.g., intro
 security vs. some other course vs.  upper-level ugrad vs. lower-level course). For example,
 we can say we are all things to all audiences, SecurityInjections may be more appropriate
 or used more heavily

 Report attendance at webinars, downloads of YouTube videos, piazza conversations, etc.

Second Pillar:
 (a) faculty survey (same instrument as SIGCSE...)
    -   * use this to drive improvement of EDURange: find disastisfied participants in early
         years, revise exercises and infrastructure along the way
 
 (b) pre/post test on a per scenario basis. Need to have identified knowledge and skills on
     a per-scenario basis in modules.tex

 

Main focus is on how well this supports faculty in their job, not necessarily a measurement
(assume, that faculty, if well-supported, will do appropriate assessment of *their* students
given educational background and institutional standards, etc.), so we will try to do pre/post
test and make available, but not main focus of our internal assessment.

Third Piller: 
 How well/completely EDURange maps to CS 2013 infosec topics.

 Timely and informed by central and guiding document of CS Ed, particularly since this document
 goes through great lengths to identify infosec topics throughout the curriculum.

 How well does EDURange support / address the need of CS 2013 to weave CS infosec throughout
 the curriculum. And the faculty are best positioned to tell us this.

 modules.tex is an assertion of the relationship between particular modules and particular
 parts of CS2013. The assessment plan here is based on how well that mapping succeeds in practice.
 This can be based on questions we ask in the faculty survey, but the *analysis* is broken out
 under this third pillar.

  * how often were you able to use an EDURange scenario as a launch pad to discuss curriculum topic X, Y, Z?

  * were there other curriculum topics, skills, or knowledge that it mapped to?



