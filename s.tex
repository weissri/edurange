\documentclass[11pt]{report}
\usepackage{times}
\usepackage{fullpage}
\usepackage{url}

\newcommand{\eat}[1]{}

\newenvironment{packenum}{
\begin{enumerate}
  \setlength{\itemsep}{1pt}
  \setlength{\parskip}{0pt}
  \setlength{\parsep}{0pt}
}{\end{enumerate}}

\begin{document}

\begin{center} 
{\Large IUSE {\em EDURange: Building Capacity for Teaching Cybersecurity Analysis Skills}}
\end{center}

The objective of this collaborative project is to disseminate interactive, competitive 
cybersecurity exercises that will be easy for computer science faculty to employ in the 
classroom, even if they do not have prior expertise in this domain.  The exercises 
will be linked to the concepts and learning outcomes described in the 
IEEE/ACM CS Curricula 2013 report, and support will be provided
to facilitate their adoption.  This support will include a student-run ``help desk'',
easy access to EDURange on Amazon's AWS/EC2, and a webinar which will include interviews
with experts who teach cybersecurity classes at the undergraduate level.  The exercises
will be accompanied by  tutorial-style instructional materials linked to learning
objectives.  The exercises themselves will provide formative assessments for the 
students.  As such, they will combine instruction and assessment and will emphasize
analytical skills.

There are three parts to our plan:
\begin{packenum}
\item Creating and improving the technology -- providing a system that is extensible, 
  easy to use, and   elastic
\item Providing faculty development --  helping them use the technology in their classrooms, 
  providing curricular resources
\item Engaging the students -- developing their skills, using their talent and knowledge, 
  and mentoring them to become the next generation of teachers and researchers.
\end{packenum}
  getting feedback on what works, and imp

%Teaching security analysis skills to undergraduates presents an
%imposing challenge for several reasons.  First, information security
%curriculum can be difficult to formulate and weave in to existing
%classes, particularly for instructors with little background in
%information security and assurance.  In addition, most existing
%security curriculum and exercises focus on teaching students basic
%secure coding practices ({\it e.g.}, input validation, bounds and
%error code checking) and lack an interactive, experiential learning
%component.  Lab exercises that {\em do} offer an experiential
%component (e.g., DamnVulnLinux, Bank) often lack configurability along
%with documentation explaining the security implications of such
%configuration choices: a major obstacle for overburdened faculty
%members without expertise in information security.  

%Such exercises need to be self-contained and easy to use.
%Google's Grugeyere is an interesting counterpoint, but only deals with
%web vulnerabilities.  
%Even if instructors do adopt existing in-depth lab exercises, the
%inherently scripted nature of such activity almost ensures that any
%such activity is unsustainable beyond a few exercises. [Needs work
%-MEL]
Almost no tools exist for helping students assess and
understand the quality (and limits) of their own security analysis
skills.  Although most principles of ``secure'' software and systems
design include advice about best practices, such activities are rarely
rewarded by the traditional academic process or even applicable to
many standard course assignements in the core CS undergraduate
curriculum.  Indeed, learning and developing the skills and principles
related to testing, debugging, analsyis, and reliability can represent
a significant investment of student time for little perceived gain. As
a result, students perceive such efforts as a distraction, and
students rarely perform any meaningful reliability or security
assessment as part of their class assignments.  Worse yet, when they do
attempt to internalize testing concepts, they often do so solely via
the use of an automated regression test suite --- although useful for
catching some unintended bugs or incompatible changes, rote
formulation of regression testing does not reward careful and
thoughtful analysis of security failure modes.
This proposal contemplates an interactive, team-based
approach to learning and practicing cybersecurity skills and
principles in an open-source, publicly available, customizable,
``live-fire'' setting.  Exercises are designed to specifically reward the time
students spend on analysis, debugging, testing, and
reverse-engineering.  

%This proposal seeks to identify ways in which we
%can construct a competition-based cybersecurity environment that has
%the right incentives to help students learn and instructors teach {\em
%  meaningful} security analysis, assurance, reliability, testing, and
%debugging skills, patterns, and principles.
% [the above sentence is a bit heavy-handed. needs judicious cutting. -MEL]

%\noindent {\bf Intellectual Merit:} Interactive information security
%scenarios provides students with a valuable tool for imparting
%practical experience with {\em analyzing failure modes, applying
%  debugging patterns, and developing assurance arguments}, but
%creating them is a labor-intensive process. In addition, exercises are
%prone to becoming rapidly outdated, and existing efforts typically
%suffer from poor documentation (or lack such documentation
%altogether).

\noindent {\bf Intellectual Merit:}
The intellectural merit of this proposal stems from two efforts.
First, EDURange exercises will address the learning outcomes of CS 2013
and reward students for understanding and
%internalizing hacker curriculum principles and approaches, 
applying analytical skills such as exercising failure modes --- 
not just learning specific tools or rote skills
such as the creation of test cases or regression tests.  

\eat{EDURange is
designed to provide students with an un-scripted active learning
environment so that students see the value of developing analysis
skills rather than training based on a pre-packaged set of tools.}

Second, the project will address a key {\em research} question:
investigating and gaining an understanding of how best to represent and explain the
security implications and semantics of a small but comprehensive set of
configuration choices (the parameter space of a scenario) to an
instructor with limited knowledge of information security.  We also
compile this representation into a working exercise.
%The lack
%of such insight and documentation is a significant flaw of existing
%cybersecurity exercises.

The research and development agenda focuses on investigating three
core issues: $(1)$ the construction of a suite of cybersecurity
scenarios; $(2)$ the deployment of this platform in a variety security
testbeds and cloud environments (at no cost to students or
instructors); and $(3)$ an understanding of how such scenarios might
be used as an evaluation and assessment tool for both students in the
course of their studies as well as an independent benchmark for other
cybersecurity training programs, curricula, and exercises.

\noindent {\bf Broader Impact:} 
This proposal seeks to develop and
deploy EDURange as an open-access/open source security scenario environment.
Several planned outcomes support the ability of this proposal to have
such an impact.  First, the successful proposal will have as an
outcome the dissemination using AWS/EC2 of exercises that have already been developed plus
additional ones.
(transparent, open-access results).  \eat{Second, these scenarios will be
implemented and hosted in a stable, widely-available infrastructure in
an existing cybersecurity testbed and in a cloud environment (specific
dissemination plan for real-world impact).}  Such wide availability
enables EDURange to serve as an independent benchmark or assessment
tool for other cybersecurity curricula.  Third, the proposal
contemplates a webinar under the auspices of Cyberwatch West
to introduce the scenarios and EDURange environment to 
instructors. 

We contemplate a significant outreach effort aimed at providing
students and instructors with a usable, configurable, and
well-documented cybersecurity scenario environment.  Our goal in this
outreach is to provide a publicly accessible environment with a
minimal barrier to entry.  EDURange helps students buy into the
process of sharpening their information security analysis skills and
makes them a partner in evaluating and understanding the limits of
those skills.

We want to change how security is taught in the classroom, what is taught, improve 
faculty confidence to teach it, and make CS 2013 a living document.  We want our students to 
have the opportunity to go beyond 
'Familiarity' to 'Usage', to be able to apply the concepts in realistic situations.



\noindent {\bf Key Words:} EDURange; cyber range; hacker curriculum
principles; teaching security failure modes; debugging patterns;
information security curriculum; live-fire cyber range exercises;
hacker bibliography.
\end{document}
