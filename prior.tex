%!TEX root = proposal.tex

\section{Qualifications and Results from Prior NSF Support}

\subsection{Richard Weiss}
%Prof. Weiss is currently funded to work on DUE: EDURange (Aug 15, 2012 - July 31, 2015).

{\bf Intellectual Merit:}
Prof Weiss had one prior NSF grant ``Collaborative Research: A Five-College Partnership for 
Information Assurance Education.''  He
worked with Mark Corner and Brian Levine 
(University of Massachusetts, Amherst),
Sami Rollins (Mount Holyoke), Scott Kaplan (Amherst College), and 
Nick Howe (Smith College), to develop an interdisciplinary undergraduate course at Hampshire College.
He worked on integrating security and computer architecture to exploit parallelism in 
cryptographic algorithms on superscalar hardware such as the DEC Alpha~\cite{rw:AES_2000}.
\eat{R.S. Weiss and N. Binkert. A comparison of AES candidates on the Alpha 21264, The Third
	Advanced Encryption Standard Conference, 2000, pp. 75–81.}


{\bf Broader Impact:}
Prof. Weiss has contributed to computer security education as an active participant in the 
SISMAT program (DUE CCLI: 0941836).  This program trains undergraduates from liberal arts colleges
in security. He was one of the core contributors to the CPATH project NWDCSD~\cite{NWDCSD}.

{\bf Ongoing Grants:}
Prof. Weiss is currently funded to work on TUES type 1: EDURange: A Cybersecurity Competition Platform to 
Enhance Undergraduate Security Analysis Skills (Aug 15, 2012 - July 31, 2015).
The goal of that project is to create cybersecurity games that could be used for learning and
assessment of security skills for students and today's security workforce.  This project is in
the middle of development, and our first multi-level 
game assesses penetration testing skills.  As part of this work,
Profs Weiss and Jens Mache have led tutorials in the last two years at CCSC-NW 
(once with Prof Vincent Nestler) 
in October~\cite{rw_jm:CCSC_2012,rw_jm:CCSC_2013}.
%``Teaching Cybersecurity Through Interactive Exercises in a Virtual Environment''
 They  also gave
a workshop at SIGCSE 2013 "Hands-on Cybersecurity Exercises and the RAVE Virtual Environment''
with Profs. Michael Locasto, Brian Hay and Vincent Nestler. Weiss, Mache, and Locasto gave another
workshop at SIGCSE 2014 ``Hands-on Cybersecurity Exercises in the EDURange Framework.'' 
 Weiss and Mache also presented a poster "Teaching Security with Interactive Exercises'' at 
ACSAC\cite{rw_jm:ACSAC_2012}, 
a paper %``Hands-On DOS Lab exercises using Slowloris and RUDY'' 
at InfosecCD~\cite{rw_jm:infosecCD_2012}, and 
will lead a BOF 
``Teaching Security Using Hands-on Exercises'' again at SIGCSE 2014. 

\subsection{Jens Mache}
{\bf Intellectual Merit:}
