\section*{EDURange Exercises}

{\bf organization and elements, learning implications}

 Targeted audience is 2-fold: (1) ``just want an exercise''/inexperienced
 (2) experienced...want configuration, etc.

%   First discuss EDURange infrastructure
\subsection{EDURange infrastructure}
EDURange allows instructors and designers to specify exercises at multiple levels of detail.
Using yaml as an intermediate representation, one can specify the structure of an exercise.
There are six types of entities that we have identified as important:
\begin{itemize}
  \item Nodes, e.g. could be a VM, attributes of the node would include players
  \item Networks,  collections of nodes and the connections among them
  \item Software, which is necessary for the correct functioning of the exercise, including versions
  \item Constraints, e.g. time limits, 
  \item Goals, i.e. learning objectives
  \item Artifact, e.g. a file that is placed on a target
\end{itemize}
Actually, the yaml file is different: groups, users, roles (an attribute), packages (Chef scripts for software)


\subsection{Existing Exercises}
\subsubsection{Recon I}
% CS2013 Curricula P.101 give the overview
Recon I  is a reconnaissance exercise, where students learn how to explore a network.
The learning goals for this exercise are:
\begin{itemize}
\item understand networking protocols (TCP, UDP, ICMP) and how they can be exploited for recon.
\item develop the security mindset
\item understand CIDR network configuration and how subdivide a network IP range.
\item use nmap to find hosts and open ports on a network.
\end{itemize}
Concepts from CS 2013 addressed by this module:
\begin{itemize}
\item  risk, threats, vulnerabilities, and attack vectors.  This exercise shows how one might 
  find vulnerabilities
\end{itemize}


This exercise is relevant for both defensive and offensive roles.  [it currently has one level]
There are a dozen hosts on a remote network, and the student tries to find all of them as quickly as 
possible, and discover what what ports are open and what services are running.  At the knowledge level,
the student is learning to use nmap and what options it uses.  At a higher level, the student is learning
about the TCP, UDP and ICMP protocols and how they can be used in ways that may have not been intended.
[more advanced: look for hosts and services that should not be on the network, knowing what to expect]

EDURange provides the capability to run the same basic scenario, changing the IP addresses and open ports
of the target hosts.  This allows students to repeat the exercise while trying different options of nmap to
find those that meet the goals of a particular variation of the exercise.  For example, the intructor
might ask students to focus on speed or stealth.

\subsubsection{Elf Infection}
The learning goals are:
\begin{itemize}
\item understand elf format
\item develop the security mindset
\item distinguish between normal and suspicious behavior.
\item use nmap to find hosts and open ports on a network.
\end{itemize}
Concepts from CS 2013 addressed by this module:
\begin{itemize}
\item  risk, threats, vulnerabilities, and attack vectors.  This exercise shows how how an attacker
  could maintain control of a system or exfiltrate data through infected binary files.
\item How does this also address trust?
\end{itemize}

\subsubsection{strace}
The learning goals are:
\begin{itemize}
\item understand how to sort through complex data
\item develop the security mindset
\item distinguish between normal and suspicious behavior.
\item use strace to characterize aspects of program behavior.
\end{itemize}


\subsubsection{scapy hunt}
The learning goals are:
\begin{itemize}
\item understand firewall rules
\item develop the security mindset
\item understand routing tables
\item understand how IP works
\end{itemize}

\subsubsection{}
The learning goals are:
\begin{itemize}
\item understand firewall rules
\item develop the security mindset
\item understand routing tables
\item understand how IP works
\end{itemize}

   a section on current Modules and their structure; this
   foreshadows the Assessment Plan. Modules include: Recon,
   ELFInfection, strace, scapyhunt, calculator grammar fuzzing,

   Description of exericse. analysis skills nurtured, cs 2013 curriculum topic.
   
   List of analysis skills.

   Modules are also an opportunistic mechanism: they provide a
   jumping-off point for discussing a variety of topics mapped to
   a -- exercise motivates each topic discussion or lesson.
