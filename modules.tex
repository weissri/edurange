\section*{EDURange Exercises}

Address immediate challenges: CS2013\\
new structures and functions: EDURange in the Cloud\\
Producing new knowledge about what works effectively.  
In addition to providing a framework for developing exercises at a high level, but we can also 
import VMs that collaborators produce.  We are putting together this service in AWS which we make available
to instructors.  This doesn't use the full power of EDURange, but rather is based on the fact that we
are providing a service through an AWS account where we can make 
The EDURange framework provides an easy way to share exercises 

{\bf organization and elements, learning implications}
With the publication of the IEEE/ACM CS2013 Curricula and its inclusion of computer security and 
information assurance, we are redirecting the exercises in EDURange to address the concepts and learning 
outcomes described in the IAS Knowledge Area.  ?Since those are still at a high level, we hope to contribute
to the implementation of the standards by providing concrete examples of those standards.  We recognize that
there are two groups of faculty who are our targeted audience: (1) ``just want an exercise''/inexperienced
 (2) experienced...want configuration, etc.

%   First discuss EDURange infrastructure
\subsection{EDURange infrastructure}
EDURange allows instructors and designers to specify exercises at multiple levels of detail.
Using YAML as an intermediate representation, one can specify the structure of an exercise.
This intermediate representation is compiled into API calls to AWS to create virtual machines and
Chef scripts to install software.  At the highest level, the description of our cybersecurity exercises
is comprised of six types of entities:
\begin{itemize}
  \item Nodes, e.g.  a VM, attributes of the node could include players with accounts on that node
  \item Networks,  collections of nodes and the connections among them
  \item Software, which is necessary for the correct functioning of the exercise, including versions
  \item Constraints, e.g. time limits, 
  \item Goals, i.e. learning objectives.  These are specified at a low level, such as 
    reading or creating an artifact, reporting an IP address or password.
  \item Artifact, e.g. a file that is placed on a target and could be used as evidence of achieving a goal
\end{itemize}

The next lower level of description is in terms of a YAML file which continas:
 groups, users, roles (an attribute), packages (Chef scripts for software)


\subsection{Existing Exercises}
\subsubsection{Recon I}
% CS2013 Curricula P.101 give the overview
Recon I  is a reconnaissance exercise, where students learn how to explore a network.
The learning goals for this exercise are:
\begin{itemize}
\item understand networking protocols (TCP, UDP, ICMP) and how they can be exploited for recon.
\item develop the security mindset
\item understand CIDR network configuration and how subdivide a network IP range.
\item use nmap to find hosts and open ports on a network.
\end{itemize}
Concepts from CS 2013 addressed by this module:
\begin{itemize}
\item  risk, threats, vulnerabilities, and attack vectors.  This exercise shows how one might 
  find vulnerabilities
\item network security
\end{itemize}


This exercise is relevant for both defensive and offensive roles.  It currently has one level of difficulty,
and we have used it in several workshops.
[show network diagram]
There are a dozen hosts on a remote network, and the student tries to find all of them as quickly as 
possible, and discover what what ports are open and what services are running.  At the knowledge level,
the student is learning to use nmap and what options it uses.  At a higher level, the student is learning
about the TCP, UDP and ICMP protocols and how they can be used in ways that may have not been intended.
[more advanced: look for hosts and services that should not be on the network, knowing what to expect]

EDURange provides the capability to run the same basic scenario, changing the IP addresses and open ports
of the target hosts.  This allows students to repeat the exercise while trying different options of nmap to
find those that meet the goals of a particular variation of the exercise.  For example, the intructor
might ask students to focus on speed or stealth.

\subsubsection{Elf Infection}
The learning goals are:
\begin{itemize}
\item understand elf format
\item develop the security mindset
\item distinguish between normal and suspicious behavior for standard utilites and programs.
\item use nmap to find hosts and open ports on a network.
\end{itemize}
Concepts from CS 2013 addressed by this module:
\begin{itemize}
\item  risk, threats, vulnerabilities, and attack vectors.  This exercise shows how how an attacker
  could maintain control of a system or exfiltrate data through infected binary files.
\item network security.  The infected software will open a port and send packets to a remote server.
\item How does this also address trust?
\end{itemize}

\subsubsection{strace}
Students are given a trace of system calls made while a few programs were running simultaneously.
The trace has been cleaned up to remove the names of the executable files.

The learning goals are:
\begin{itemize}
\item understand how to sort through complex data
\item develop the security mindset
\item distinguish between normal and suspicious behavior.
\item use strace to characterize aspects of program behavior.
\end{itemize}
Concepts from CS 2013 addressed by this module:
\begin{itemize}
\item  risk, threats, vulnerabilities, and attack vectors.  

\item How does this also address trust?
\end{itemize}


\subsubsection{scapy hunt}
In this exercise, students craft network packets to detect hosts that are not visible directly on
the local network.  Describe listening for packets and replaying them with modifications, overflowing
the forwarding table, port knocking.

The learning goals are:
\begin{itemize}
\item understand firewall rules
\item develop the security mindset
\item understand routing tables
\item understand how IP works
\end{itemize}


\subsubsection{Grammar Fuzzing}
This is an attack/defend exercise, where students must implement a recognizer for a grammar,
e.g. a grammar describing the input to a calculator.  The attacker can look at the code and
must design input that either creates false postives or false negatives.

The learning goals are:
\begin{itemize}
\item perform input sanitization
\item develop the security mindset
\item understand langsec
\end{itemize}

\subsection{Addressing CS2013}
Here are the concepts and learning outcomes from CS2013 that we will address:


Here are additional exercises that will address those goals:


\comment{   a section on current Modules and their structure; this
   foreshadows the Assessment Plan. Modules include: Recon,
   ELFInfection, strace, scapyhunt, calculator grammar fuzzing,

   Description of exericse. analysis skills nurtured, cs 2013 curriculum topic.
   
   List of analysis skills.

   Modules are also an opportunistic mechanism: they provide a
   jumping-off point for discussing a variety of topics mapped to
   a -- exercise motivates each topic discussion or lesson.}
