\subsection*{Related Work}
\label{ssec:related}
\comment{
Hinted at these problems in the intro, but give a more fulsome description of each, plus the specific difference
/ advantage of edurange. Best way to do this is emphasize different focus and outcomes? “We’re not seeking
to replace, but to build/augment capacity along another direction.”
}

EDURange provides a unique infrastructure that enables faculty to develop exercises for students, in which they
interactively develop their information security analysis skills.
Information security education has been a popular, well-studied topic.
Preparing a large cybersecurity workforce seems to be a national
priority, but emerging viewpoints~\cite{cooper2009sigcse} seem to
suggest that most efforts to date have not been as effective as
government and industry seems to need~\cite{locasto2011cacm}. ** is this still true? **
GovTech.com suggests that there is a ``lack of faculty at the
university level who can teach cyber-security beyond its ``soft side''
including policy and
analysis.''\footnote{\url{http://www.govtech.com/security/Cyber-Challenge-Work-Force-041811.html}}


We believe that we have identified an interesting and previously
underexplored approach to enabling students to understand and apply
principles and patterns (particularly dealing with the composition of
trust).  The challenge that we are facing is how to make it more broadly available to
faculty and their students.

\eat{rather than learn only toolsets or stale, ten-year old attacks
and exploits.}

Judging from the Birds of a Feather sessions on Security at the last two years of SIGCSE, there
has been a modest increase in the number of hands-on cybersecurity exercises, not all of which
address the security mindset.  It is also clear from the workshops we have attended that faculty 
want to use them but are frustrated because of the difficulty in disseminating and setting them up.
The other frameworks that we have tried have some good features that we can emulate and some limitations
that we try to avoid

\eat{There are a number of hands-on exercises that are available 
Tennessee Tech University
UW Tacoma Yan Bai, VMs on flash drives 
ASIDE (Bill Chu)}

2. Frameworks for Security Labs \\
DETER, werewolves, RAVE
These environments are fine, except that with RAVE, new scenarios and VMs need to be installed by
the developers.  A faculty member can work with the developers to create new configurations an exercises,
they can't just upload some VMs and try them out.

Currently there are a few frameworks for 

\subsection{Comparison of EDURange with Existing Projects}

As we detail in Table~\ref{table:compare}, EDURange provides several
unique and distinct experiences from existing cybersecurity labs,
exercises, and curricula.  The PacketWars project is probably the
closest existing piece of work to what EDURange proposes, and there
has been some work on providing students with access to ``live''
exercises on a small scale~\cite{vigna}.
\comment{
\begin{table*}[ht]
\caption{{\em A Summary Comparison of EDURange and Existing Projects.} 
EDURange focuses on developing analytical skills and understanding system and network failure modes. The table below is not a criticism of existing efforts, but meant to highlight the ways in which our plans for EDURange differ from the characteristics of existing projects -- these projects may have been built with different criteria in mind.}
\begin{center}
\begin{tabular}{|l||c|c|c|c|c|c|}
\hline
 {\bf Project} & {\bf Analysis} & 
                 {\bf Competition} &
                 {\bf Configurable} & 
                 {\bf Scalable} & 
                 {\bf Documentation} &
                 {\bf Coverage}\\
\hline
 EDURange &  %\checkmark & \checkmark & \checkmark & \checkmark & \checkmark & \checkmark
\\
\hline
 Cybersiege & \\
\hline
 SEED & \\
\hline
 Security Injections & \\
\hline
 CCDC & \\
\hline
 PacketWars & \\
\hline
ITSEED & \\
\hline
 Google Gruyere & \\
\hline
Tennessee &\\
\hline
\end{tabular}
\end{center}
\label{table:compare}
\end{table*}
}

\begin{table*}[ht]
  \caption{{\em A Summary Comparison of EDURange and Existing Projects.}
      EDURange focuses on developing analytical skills and understanding
      system and network failure modes. The table below is not a criticism
      of existing efforts, but meant to highlight the ways in which our
      plans for EDURange differ from the characteristics of existing
      projects -- these projects may have been built with different
      criteria in mind.}
  \begin{center}
    \begin{tabular}{|l||c|c|}
    \hline
    {\bf Project} & {\bf Primary Disadvantage} & {\bf Best Feature} \\
    \hline
     Cybersiege & shallow analysis
                & interactive training scenarios\\
    \hline
     SEED & lacks competitive interaction
          & comprehensive documentation\\
    \hline
     Security Injections & focus on defensive coding patterns
                         & introduction to basic security\\
    \hline
     CCDC & requires travel; limited remote access
          & interactive and competitive\\
    \hline
     PacketWars & requires travel; limited availability
                & contains well-structured scenarios\\
    \hline
    ITSEED & minimal instructor support, distrib by flash drive
              & good documentation for students\\
    \hline
     Google Gruyere & narrow focus (web apps)
                    & cloud-based; well-documented\\
    \hline
    Security Knitting Kit & not distributed & ?? \\
    \hline
    The RAVE & not very flexible & cloud-based; existing lab manual \\
    \hline
    Seattle Testbed & limited in scope & easy-to-use; well-documented; P2P \\
    \end{tabular}
  \end{center}
  \label{table:compare}
\end{table*}



Some of the projects listed in Table~\ref{table:compare} seem to have
different educational goals in mind.  For example, while Cybersiege
from Naval Postgraduate School provides a video-game-like, interactive
environment, the decisions asked of a participate are mainly
requisition and provisioning decisions rather than problems requiring
technical analysis skills; this difference speaks to a different
training goal.  Google's Gruyere is a great example of an interactive,
well-documented, scalable cybersecurity exercise, but it only covers
the space of web application vulnerabilities.  Towson's ``Security
Injections'' mainly focus on several important security programming
patterns.  While the SEED material from Syracuse presents a mature,
well-documented set of exercises, they are not typically interactive
or dynamic.  Competitions like CCDC seem to be structured in such a
way as to require travel to the competition; EDURange envisions a
publicly-accessible platform where even remote parties can interact.
The RAVE is set up for a specific set of exercises, which are not interactive
according to our definition.  It is provides good isolation, so that it would be very difficult
for a student to gain access to the Internet from the RAVE environment.  On the other hand,
it limits the insructors in the same way, making it impossible to modify the configuration 
except through a request to the RAVE developers.

\subsection{Information Security Education}

Our philosphy on information security education is based on our work
on understanding and teaching the hacker
curriculum~\cite{bratus2007hacklearn} to students.  This approach is
predicated on the utility of failure modes --- rather than teaching
students the ``success'' cases, we attempt to deliver a culture shock
that makes them disrespect API boundaries and adopt a cross-layer view
of the CS discipline~\cite{bratus2010sigcse}.  We routinely encourage
our students to adopt a dual frame of mind when solving
problems~\cite{locasto2009own} to prevent artificial abstraction
layers from becoming boundaries of competence~\cite{white1996csedsec}.

%http://www.nsf.gov/awardsearch/showAward.do?AwardNumber=0817267






