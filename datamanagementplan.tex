\section{Data Management Plan}

{\bf Data Type:} 
We will capture survey information using Qualtrics (which our evaluation 
consultant has prior experience with).   

%Additionally, for students that opt-in, we will also capture 
%information about how they use Seattle via a plug-in to the service manager.   
%This information will consist of the commands typed in the service manager, 
%files uploaded to Seattle nodes, and similar items.   It will not capture
%any data other than those that are input into the Seattle service manager.

The educational modules that are created as part of this work will be
disseminated through Ensemble.  This resource is funded by NSF and used by instructors 
to find educational modules.
We will also cross-post and link these modules on www.edurange.org.
We will publicly post all code on github.com.

{\bf Evaluator Data Access} We will share in a timely manner, all the 
experimental data and measures we collect with Dr. Nilsen
so that he can provide help and oversight to synthesize our results into an 
ongoing report.  Such access serves the critical purpose of keeping him
current with the PI's data gathering and analysis activities so that they can 
continually update and improve educational modules and supporting materials.
Personally identifiable information will be remove from survey data before it is stored.

{\bf Legal and Ethical Issues:} No specific ethical or legal issues 
are anticipated to arise in this work.   
%Our testbed restricts the set of 
%things a researcher can do to prevent things like source address spoofing
%and ICMP packets (both are major sources of complaints on PlanetLab).
%While we have never had cause to use it in practice, we have implemented an 
%emergency stop mechanism to use in response to abuse complaints.   

{\bf IRB Preparation:} Our activities in this proposal do not need any
specific IRB approval.   %are in the process of designing an IRB protocol to 
%encompass our assessment activities because they involve a student to opt-in
%to allow evaluators to observe their interaction with Seattle.  We
%will submit this protocol to our respective IRBs before the planned start date 
%of the project. 
The PIs and other listed personnel have previously completed 
IRB and human studies ethics training.

{\bf Access, Data Sharing, Reuse:} Any code from this project will be released 
under the open source MIT license.
Project results will be widely 
disseminated to the community through project website, conference publications,
journal articles, NSF annual meetings, annual reports, and similar methods.
Dissemination to general public will be achieved through media
outlets, magazine publication, presentation in public forums, and on www.edurange.org.

{\bf Data Standards and Capture Methods:} 
To collect survey data, we will use Qualtrics.  The consultant has experience
collecting and analyzing survey data using this platform.  Data can also be 
exported from this platform into Excel for easy manipulation or exfiltration
by other parties if desired.

We will continue to use standard development tools like Basecamp and git to manage 
educational modules, code, document the project, and synchronize effort across 
participants. 


{\bf Short-term and Long-term Data Storage and Data Management:} 
To date, the amount of data stored by this work can fit on a standard 
server.  Although the anticipated data volume is hard to evaluate, we 
estimate that a 2TB drive will easily suffice. 

No specific security measures will be needed 
concerning data storage. 

\begin{enumerate}
\item System and measurement data will be stored and kept online for three 
years beyond the project life.
\item Source code will be stored on both local file servers and github.com.  These are
adequate for long-term archiving.
\item Electronic copies of reports and articles will be kept on multiple 
devices, such as cloud document storage, laptops, desktops, and servers.
\end{enumerate}

{\bf Resources:}
The PI will supervise and be responsible for implementing and maintaining this 
data management plan.  Senior personnel, students, and contributors
will be given a briefing on this plan prior to obtaining write access to github
or Basecamp.  This will ensure a smooth implementation of this plan.
